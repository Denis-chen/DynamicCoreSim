Mobile computing devices have grown to require processors that can support a dynamic workload. The typical serial tasks such as voice compression, speech transcoding (for communications), and image compression must perform well. However, newer mobile devices need to render complex graphics and run advanced background scheduling, all while maintaining serial performance and power. The obvious answer might be GPUs, but they are known to be power-hungry chips. Ideally, if most of the serial and parallel tasks could be performed on a single chip, the GPU would only need to be used to perform high thread level parallelism tasks like displaying graphics. A smaller GPU workload means a lower power GPU. Thus, power consumption can be kept minimal, while performance gains are still attained.

Dynamic core processors attempt to solve precisely these issues. By adding an interconnection network, some additional hardware, and control logic, the symmetric multicore embedded processors can be reconfigured into a modern, super-scaler, out-of-order processor. Thus, by identifying serial and parallel program sections, the processor can be dynamically changed to perform efficiently.

Obviously, creating such a processor, though possible, is not a trivial task. So, we attempt to analyze the theoretical performance gains of a dynamic core as described above. The following work will show whether a dynamic core processor performs better than a symmetric multicore processor and a modern, super-scalar, out-of-order processor. Furthermore, for a fixed length program, we determine how often the program must switch from serial to parallel processing before a dynamic core processor realizes performance gains. Finally, we determine the maximum number of cores in a dynamic processor before critical path length negatively impacts serial performance.