The motivation for this paper came from the paper Amdahl's Law in the Multicore Ere \cite{hill}. Here Amdahl's law was extended to cover multiple multiprocessing hardware models based on a limited budget of hardware resources, or BCEs (Basic Core Equivalents). Based on both the percent of the workload that is parallelizable, the distribution of BCEss, and how performance of a single core scales, these give new speedup formulas to model system level hardware. The models used were Symmetric - same BCEs per core, Asymmetric - one large core and the rest small, and Dynamic - all BCEs switch between one large core and numerous small. 

From there the curves were displayed for various total BCEs and fractions parallel, deriving results to consider for designing multicore systems. Among these results are that parallelism is still critical for achieving good speedups, increasing performance per core is globally efficient even if locally inefficient for symmetric and asymmetric. Furthermore, dynamic systems are better than asymmetric, which are better than symmetric. Based on this result for a dynamic system offering superior performance gain, we decided to investigate dynamic multiprocessing as an avenue to improve performance.
